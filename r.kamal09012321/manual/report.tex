\documentclass[a4paper,10pt]{report}
\usepackage[utf8]{inputenc}
\usepackage{amsthm,amssymb,mathrsfs,setspace,textcomp,amsmath}
\usepackage{amsmath}
\usepackage{setspace}
\usepackage{epsfig}
\usepackage{geometry}    
\usepackage{graphicx}
\usepackage{graphics}
%\usepackage{comment}
\usepackage{float}
\restylefloat{figure}
%\usepackage[active]{srcltx}
%\usepackage[notcite,notref,color]{showkeys}
%\usepackage{epstopdf}
%\usepackage{graphicx, subfigure}
%\usepackage{epsfig}
\renewcommand{\chaptermark}[1]{\markboth{#1}{}}
\renewcommand{\sectionmark}[1]{\markright{\thesection\ #1}}


% Title Page
\title{Report on K-Mean Algorithm Implementation for Discrete Cover problem }
\author{Raj Kamal \\
        Roll no :-09012321}


\begin{document}
\maketitle

% --------------- Abstract page -----------------------
\begin{center}
{\Large{\bf{ABSTRACT}}}
\end{center}
The purpose of this report is to give manual of the continuos and discrete version of k-MEAN algorithm.
We have used the Continuos and Descrete K Mean Algorithm to approximate the solution of Descrete Square Unit Cover Problem.Given 
n red point and m blue point.We have the red points using blue points.A red point is said to be covered if we draw
a circle of radius rad center around a blue point out of blue point set ,then that red point must lie inside the circle.
We will use Descrete version of K Mean algorihtm to arrive to the answer.

\clearpage

\tableofcontents
\chapter{Introduction}
\begin{itemize}
 \item \textbf {Header Files}
      \begin{enumerate}

      \item \textbf{iostream.h} \\
           This is standard $C++$ header file for input and output operations.      
      \item \textbf{cstdio}\\
           This is standard $C++$ header file for using functions defined in C
      \item \textbf{fstream.h} \\
           This is standard $C++$ header file for  file operations
      \item \textbf{cstlib}\\
            This is standard $C++$ header file for using exit() functions
      \item \textbf{cmath}  \\
            This is standard $C++$ header file for mathematical functions like sqrt() square root.
      \item \textbf{vector}  \\
            This is standard $C++$ header file for implementing vectors and vector related functions.Vectors are dynamic
            array.
      \item \textbf{glu.h glut.h glui.h}\\
            This is standard OpenGl Header files for implementing GUI .
\end{enumerate}

\item \textbf{User Defined Header File} \\
            $user\_def.h$ is user defined library file .All Global varibles, Function Prototypes and User defined Functions are
            listed here.All parameters like height,width of frame are listed here.Any ways definations are explicitly 
            defined in other files.
 
\item \textbf{Generating Functions}        \\
            $algo\_fun.cpp$ contains fucntions relating to generating points.All different generating fucntions are
            defined here.

\item   \textbf{Algorithm Functions}          \\
            All algorithm related functions defined in $algo\_fun.cpp$.Like creating cluster, changing median , creating actual 
            cluster.

\item  \textbf{Display Functions}              \\
           All algorithms related to display are defined in $diplay\_fun.cpp$.Like displaying points, displaying actual cluster
           and all.

\item   \textbf{KeyBoard Functions}         \\
          Several keys are used for implememting functions.ALl these are defined in $keyboard\_fun.cpp$.All the important
          keys are displayed to console.
          
\item \textbf{Mouse Functions}            \\
          Mouse left and middle buttons are used for implementing functions that are defined in $mouse\_fun.cpp$.It includes
          functions like drawing red point and blue points.Also functionality associated with right click are defined in main.cpp
          due to some technical issues.
       
\item \textbf{Main Function}              \\
          Like every $C++$ project contains  one main function, that main function is defined in $main.cpp$.It also
          contains definations of GLobal variables which can be set here.Also it contains the functionality definations
          of right click of mouse button.
          
          
\item \textbf{menu Function }              \\
          It displays the menu at console and defined in $menu.cpp$
          
\end{itemize}

\chapter{Code }
\begin{itemize}
 \item Install some of the libraries \\
       \textbf{sudo apt$\textendash$get install freeglut freeglut3$\textendash$dev g$++$} 
 \item The code is made using C and C$++$ libraries.
 \item It implements the Continuos K Mean Algorithm and Descrete K mean algorithm in background and display the results.
 \item Several Global Variables are defined in $main.cpp$ file where any user can put their value
 \item There are several functionality with keyboard and mouse .Any time program hang then its the case see
       the command promot ,it must be asking for input and if exited we can see $error.txt$ to see the error.
\end{itemize}
\clearpage
\section{Continuos K Mean Algorithm}
\subsection{Usage}
Initially A blank screen will appear and menu on terminal.Any option written in menu in terminal will work when it
is pressed while output screen window is open.We can generate points and then press \textbf{t} to get result.
\begin{itemize}
 \item Press \textbf{esc} to quit the program.
 \item Press \textbf{0} to display the menu again.
 \item Press \textbf{Left mouse} to draw red point
 \item Press \textbf{Middle mouse} to draw blue point
 \item Press \textbf{Right mouse} button to explore further option.
 \begin{enumerate}
  \item \textbf{Generate} .It opens another menu and pressing red blue or cluster will generate random red blue cluster.It will
                 require input which will be taken at terminal.
  \item \textbf{Display} .It opens another munu and pressing the feild will display that in termimal or output window.
  \item \textbf{Set}.It helps to set the varibles used in program.
  \item \textbf{Exit} ,It will cause program to quit.
 \end{enumerate}
 \item We can generate specific number of red point at arbit position by pressing \textbf{r} while output window is open.
 \item Similary we can generate blue points by pressing \textbf{b}.
 \item We can generate both red and blue points at once by pressing \textbf{a}.\\
  \textbf{NOTE} we can set domain by changing upper and lower limits in $main.cpp$.
 \item We can check the feasiblity by pressing \textbf{F}.It will display on terminal whether the points are feasible
 or not feasiblity.
 \item We can output complete results with the red points and blue points by pressing \textbf{t}
 \item we can output results with the present cluster size keeping same red and blue points by pressing \textbf{I}
 \item we can output results with the incremented cluster size keeping same red and blue points by pressing \textbf{i}\\ \\
 \textbf{NOTE} We can get result in one step by pressing \textbf{t}. But if we want to see how it is actually showing
 resutls with incrementing cluster size for that first press \textbf{I} and then keep pressing \textbf{i} and see the
 results step by step.ALso at any point of time if you want to see the ouptut with present clsuter size press \textbf{I}.
 \item We can see the point and cluster by pressing \textbf{A}.
 \item we can see the Blue points only by pressing \textbf{B}.
 \item we can see the red points only by pressing \textbf{R}.
 \item we can see the cluster only by pressing \textbf{C}.
 \item we can see the actual cluster by eliminating reduntant cluster by pressing \textbf{D}.
 \item we can see the blue cluster,(contributing to result) corresponding to clsuter algorithm found by pressing \textbf{E}.
 \item we can see the final cover or result by pressing \textbf{G} \\ \\
 \textbf{NOTE} Here algorithm found cluster at position where blue point are not situated.Intially it found cluster with 
 the set value which we can see by pressing \textbf{C}. In that there may be reduntant cluster we can see the result after
 removing them by pressing \textbf{D}.Now the clsuter head may be not at blue point so we search for corresponding blue points 
 which we can see by pressing \textbf{E} and finaly we can see results by pressing \textbf{G}.Here only results that have
 been produced by pressing \textbf{i,I ,t} are shown.It does not run any algorithm.It just display.
 \item we can see how many red points are generated by pressing \textbf{1}.
 \item we can see how many blue points are generated by pressing \textbf{2}.
 \item we can see how many cluster are there by pressing \textbf{3}.
 \item we can see how many cluster out of cluster resulted by algorithm participated in coverring by pressing \textbf{4}.
 \item we can see number of blue point reqiuired to cover by pressing \textbf{5}.
 \item We can get complete result by pressing \textbf{6}.\\ \\
 \textbf{NOTE} we can see results in terminal only.
 \item we can regenerate same number of points and then display the result by pressing \textbf{h}.It is usefull when we 
 are analying results for fixed number of points.we can get an idea about how many points in the average requird to cover.
 \item We can change cluster value and display the result by pressing \textbf{K,L,M}.
 \item We can do statisical analysis by pressing \textbf{S}.Here algorithm is run for 100,200,500,700,1000 red point.
 blue points are generated and cover is found and result is saved in file $result.txt$,other analysis are stored in
 $data.txt$ and $convergence.txt$.We can change the number of red point by opening $keyboard_fun.cpp$ and changing line
 374 and 376.Here input is asked for upper and lower bound for red and blue points..
 \end{itemize}
 

 \textbf{NOTE} For the first time pressing left mouse may not display red point but as middle mouse button 
 is pressed , all points drawn can be seen.It happens for first time only.



\clearpage

\clearpage
\section{Descrete K Mean Algorithm}
\subsection{Usage}
Initially A blank screen will appear and menu on terminal.Any option written in menu in terminal will work when it
is pressed while output screen window is open.We can generate result  and press \textbf{r} to get result.
\begin{itemize}
 \item Press \textbf{esc} to quit the program.
 \item Press \textbf{m} to display the menu again.
 \item Press \textbf{Left mouse} to draw red point
 \item Press \textbf{Middle mouse} to draw blue point
 \item We can clear the screen by pressing \textbf{A}.
 \item We can generate specified number of red and blue points at random places by pressing \textbf{w}.
 \item We can generate specifeid number of red point at random places by pressing \textbf{B}
 \item We can generate specified number of blue point at random places by pressing \textbf{R}.\\ \\
 \textbf{NOTE} we can set domain by changing upper and lower limits in $main.cpp$.
 \item we can check feasiblity by pressing \textbf{r}.Result will be displayed on terminal.
 \item We can output complete results with the red points and blue points by pressing \textbf{t}
 \item we can output results with the present cluster size keeping same red and blue points by pressing \textbf{I}
 \item we can output results with the incremented cluster size keeping same red and blue points by pressing \textbf{i}\\ \\
 \textbf{NOTE} We can get result in one step by pressing \textbf{t}. But if we want to see how it is actually showing
 resutls with incrementing cluster size for that first press \textbf{I} and then keep pressing \textbf{i} and see the
 results step by step.ALso at any point of time if you want to see the ouptut with present clsuter size press \textbf{I}.
 \item we can show red points and blue points  by pressing \textbf{a}.
 \item we can show red points only by pressing \textbf{j}.
 \item we can show blue points only by pressing \textbf{b}.
 \item we can display cluster points by pressing \textbf{c}.
 \item we can display actual clsuter point that is cover by pressing \textbf{d}.
 \item we can show results on terminal by pressing \textbf{4}.
 \item we can show number of red points, blue points and cover size by pressing \textbf{1,2,3} respectively.
 \item we can ulter K and incrementing factor by pressing \textbf{k,l,K}
 \item we can enter new radius by pressing \textbf{N}.
 \item we can show results by maintaining same cluster by pressing \textbf{u}
 \item we can shoe results by maintaining same cluster size but different cluster by pressing \textbf{v}.\\ \\
 \textbf{NOTE}The option \textbf{u,v} are given for analysis purpose.we can get idea about chossing differnt cluster how 
 result is going to get affected.
 \item For statistical analysis purpose we press \textbf{e}.we can details of it in line 121 of $keyboard_fun.cpp$.All results
 are displayed in $result.txt,data.txt$ and analyis part in $convergence.txt$.
 \end{itemize}
 \textbf{NOTE} For the first time pressing left mouse may not display red point but as middle mouse button 
 is pressed , all points drawn can be seen.It happens for first time only.
 
\end{document}          